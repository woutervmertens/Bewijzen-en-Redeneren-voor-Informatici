\documentclass{article}
\usepackage{graphicx}
\usepackage[utf8]{inputenc}
\usepackage{amssymb}
\usepackage{hyperref}
\graphicspath{ {images/} }

\title{Bewijzen en Redeneren voor Informatici}
\date{2019 - 2020}

\begin{document}

\maketitle
\tableofcontents

\section{Verzamelingen}

Aantal veschillende elementen behandeld als 1 ding: K = \{a,b,...\}
Geschreven als opsomming : \{a,b,c\}, of omschrijving : \{ x \big| x is een klinker \} met x een variabele.

\begin{itemize}
\item $\emptyset : leeg , \{a\} : singleton , \{a,b\} : paar, \mathbb{N} : oneindige verzameling$
\item Intervallen: \newline
    \{ x $\in \: \mathbb{R} \: \big| \: 0 < x < 1 \} = \: ]0,1[ \: $: open \newline
    \{ x $\in \: \mathbb{R} \: \big| \: a \leq x \leq b \} = [a,b] \: $: gesloten \newline
    \{ x $\in \: \mathbb{R} \: \big| \: x \leq a \} = \: ]-\infty,a] \: $: oneindig
\item Deelverzamelingen: \newline
    A $\subseteq$ B : A is een deelverzameling van B $\Rightarrow$ "$\forall x \: \in \: A$ geldt: $x \: \in \: B$" \newline
    A $\subset$ B : A is een strikte deelverzameling van B $\Rightarrow$ A $\neq$ B
\end{itemize}

\subsection{Operaties op verzamelingen}
\begin{itemize}
    \item doorsnede: A $\cap$ B = \{ x \big| x $\in$ A en x $\in$ B \}
    \item unie: A $\cup$ B = \{ x \big| x $\in$ A of x $\in$ B \}
    \item verschil: A $\setminus$ B = \{ x \big| x $\in$ A en x $\notin$ B \}
    \item complement: A\textsuperscript{C} = \{ x \big| x $\notin$ A\} = U $\setminus$ A (U = universum)
    \item symmetrisch verschil: A $\bigtriangleup$ B = (A $\setminus$ B) $\cup$ (B $\setminus$ A)
\end{itemize}
A en B zijn disjunct: A $\cap$ B = $\emptyset$ \newline
A $\setminus$ B = A$\setminus(A \cap B)$ = $A \cap B\textsuperscript{C}$ \newline
associatief: volgorde operaties maakt niet uit \newline
commutatief: volgorde verzamelingen maakt niet uit: $A \cup B = B \cup A$ \newline
distributief: operatie te verdelen binnen haakjes: $A \cup (B \cap C) = (A \cup B) \cap (A \cup C)$ \newline $\Rightarrow A \cap B \cap A\textsuperscript{C} = B \cap (A \cap A\textsuperscript{C}) = B \cap \emptyset = \emptyset$ \newline
$\bigcup_{i=1}^{n} A_{i}$ = unies $A_{1}$ tot $A_{n}$ (zelfde met $\bigcap$) \newline

\subsection{De Morgan}
\begin{itemize}
    \item $(A \cup B)^C = A^C \cap B^C$
    \item $(A \cap B)^C = A^C \cup B^C$
    \item $A \cup A = A = A \cap A$
    \item $A \setminus A = \emptyset$
\end{itemize}

\subsection{Machtsverzamelingen}
De machtsverzameling van A: P(A), is de verzameling van alle deelverzamelingen van A. \newline
    $A = \{1,2\} \Rightarrow P(A) = \{\{1\},\{2\},\{1,2\},\emptyset\}$ \newline
aantal elementen van A = $|A|$ = \#A = 2, \#P(A) = 4

\section{Precieze uitspraken}
Doel: dubbelzinnigheden vermijden $\Rightarrow$ wiskundige notatie gebruiken
\begin{itemize}
    \item Conjunctie: $P \land Q$ : en
    \item Disjunctie: $P \lor Q$ : of (inclusief)
    \item Negatie: $\neg P$ : niet
    \item Implicatie: $P \Rightarrow Q$ : P dan Q
    \item Equivalentie: $P \iff Q$ : P asa Q
\end{itemize} 
Prioriteit: $"\neg" > "\land" > "\lor" > "\Rightarrow / \iff" $, of haakjes zetten \newline
Logische equivalentie: $\equiv$ : heel de waarheidstabel: alle lijnen moeten hetzelfde zijn
\begin{itemize}
    \item Tautologie: altijd waar: vb $P \lor \neg P$
    \item Contradictie: altijd onwaar: vb $P \land \neg P$
    \item $\neg (\forall x \in A: P(x)) = \exists x \in A: \neg P(x)$
    \item $\neg (\exists x \in A: P(x)) = \forall x \in A: \neg P(x)$
\end{itemize}
predikaat P, P(x) is bewering \newline

Stel: S is een verzameling, B(x) beweert $x \in S$, P(x) is willekeurige bewering voor x:

    \[\forall x \in S: P(x) \longrightarrow \forall x: B(x) \Rightarrow P(x)\]
    \[\exists x \in S: P(x) \longrightarrow \exists x: B(x) \land P(x)\]

$P \Rightarrow Q \equiv \neg P \lor Q$ \newline

\begin{center}
    \begin{tabular}{|c|c|}
    \hline
    logisch & verzameling \\
    \hline
    $\lor$ & $\cup$\\
    $\land$ & $\cap$\\
    $\neg$ & \textsuperscript{C}\\
    \hline
    \end{tabular}
\end{center}
Hieruit volgt dat de logische ook commutatief en associatief zijn en de regels van De Morgan volgen, want:
\begin{itemize}
    \item $A \cup B = \{x \big| x \in A \lor x\in B\}$
    \item $A \cap B = \{x \big| x \in A \land x\in B\}$
    \item $A\textsuperscript{C} = \{x \big| \neg(x \in A)\}$
\end{itemize}
dus De Morgan:
\begin{itemize}
    \item $\neg(P \lor Q) \equiv \neg P \land \neg Q$
    \item $\neg(P \land Q) \equiv \neg P \lor \neg Q$
\end{itemize}
\section{Opbouw van theorieën}
Opgebouwt met:
\begin{itemize}
    \item definities
    \item notatie-afspraken
    \item eigenschappen
    \item axiomas: definiërende eigenschappen
    \item stellingen: niet bewezen
    \item bewijzen
    \item lemmas = hulpstellingen
\end{itemize}

\section{Bewijzen}
Context/nut:
\begin{itemize}
    \item correctheid van algoritmen
    \item eindigheid van algoritmen
    \item efficiënte oplosbaarheid/complexiteit van problemen
\end{itemize}
Soorten
\begin{itemize}
    \item per Vaststelling: $(P \Rightarrow Q) \equiv (\neg P \lor Q)$ : waarheidstabellen uitschrijven: zijn gelijk
    \item per Constructie (speciale soort Vaststelling) : $\exists x: P(x)$ : vind een x waarvoor het klopt
    \item per Tekening (speciale soort Vaststelling) : teken beide zijden van equivalentie  
\end{itemize}
Bewijsstappen (1 stap):
\begin{itemize}
    \item substitutie:
        \begin{itemize}
            \item voor een universeel gekwantificeerde variabele ("voor alle verzamelingen X") 1 waarde invullen: $\forall \rightarrow \exists$
            \item een uitdrukking vervangen door een equivalente uitdrukking
        \end{itemize}
    \item gebruik de definities: vervormen naar iets dat al bewezen is/onmogelijk is
    \item modus ponens: als P waar is en P impliceert Q waar, dan is Q waar
\end{itemize}
Bewijsstrategieën:
\begin{itemize}
    \item ketens van gelijkheden/ongelijkheden/equivalenties/implicaties: op basis van transitiviteit
    \item wederzijdse implicaties : $P \Rightarrow Q \land Q \Rightarrow P \rightarrow P \iff Q$
    \newline variant: wederzijdse inclusie: A = B via $A \subseteq B \land B \subseteq A$
    \item herhaalde gevolgstrekking: meerdere modus ponens
    \item gevalsonderscheiding: verschillende mogelijkheden apart beschouwen
    \item in het ongerijmde: als niet waar dan is iets wat niet waar is waar $\Rightarrow$ waar
    \item door inductie
\end{itemize}

\section{Relaties}
\begin{itemize}
    \item tupel: n objecten $x_{1},x_{2},...,x_{n}$ als 1 object ($x_{1},x_{2},...,x_{n}$)
    \item koppel: 2-tupel: vb (x,y)
    \item Cartesisch product: $A \times B = \{(x,y) | x \in A \land y \in B \}$ 
    \newline vb: A = $\{S,M,L\}, B = \{z,w\}$
    \newline $A \times B = \{(S,z),(S,w),(M,z),(M,w),(L,z),(L,w)\}$
    \newline $A_{1} \times A_{2} \times ... \times A_{n} = \{(x_1,x_2,...,x_n)|x_1 \in A_1 \land ... \land z_n \in A_n \}$
    \newline $A^1 = A \quad en \quad \forall n > 1 : A^n = A^{n-1} \times A$
    \item Relaties: verband tussen 2 of meer dingen
    \begin{itemize}
        \item relatie tussen A en B is deelverzameling van A $\times$ B
        \item notatie: R(x,y)
        \item grafisch diagram A en B met pijlen van x naar y
    \end{itemize}
    \item Inverse relatie: $R^{-1} = \{(x,y)|(y,x)\in R\}$
    \item Samengestelde relatie = samenstelling: $S \circ R$ (= "S na R") = $\{(x,y)|\exists x \in B:(x,z)\in R \land (z,y)\in S\}$ met $R\subseteq A \times B$ en $S\subseteq C \times D$
    \begin{itemize}
        \item voor alle binaire relaties R,S,T: T $\circ$ (S $\circ$ R) = (T $\circ$ S) $\circ$ R
        \item R binaire relatie: $R^1$ = R en $\forall n > 1: R^n = R \circ R^{n-1}$
    \end{itemize}
\end{itemize}

\section{Functies}
\begin{itemize}
    \item relatie $R \subseteq A \times B$ is functie A naar B asa voor elke x $\in$ A hoogstens 1 (x,y) $\in$ R bestaat.
    \begin{itemize}
        \item y: beeld van x
        \item A: bronverzameling
        \item B: doelverzameling
        \item $\forall x \in A: (x,y) \in R \land (x,y') \in R \Rightarrow y = y'$
    \end{itemize}
    \item notatie:
    \begin{itemize}
        \item f functie A naar B: f:A$\rightarrow$B
        \item f beeld x af op E: f:x$\mapsto$E
        \item $\Rightarrow f:A \rightarrow B: x \mapsto E$
        \item beeld van x onder f: f(x)
    \end{itemize}
    \item domein van f: A $\rightarrow$ B is $\{x| \exists y: (x,y) \in f\}$
    \item bereik van f: A $\rightarrow$ B is $\{y| \exists y: (x,y) \in f\}$
\end{itemize}

\subsection{Afbeeldingen}
\begin{itemize}
    \item $f: A \rightarrow B$ is een afbeelding van A naar B asa voor elke x $\in$ A precies 1 (x,y) $\in$ f bestaat
    \item afbeelding $f: A \rightarrow B$ is een surjectie asa elke y $\in$ B het beeld is van x $\in$ A
    \item afbeelding $f: A \rightarrow B$ is een injectie asa elke x $\in$ A op een verschillende y $\in$ B afgebeeld wordt
    \item \quad $\Rightarrow$ surjectie: elke y gebruikt, soms door meerder x. injectie: elke x naar 1 andere y, niet per se alle y gebruikt
    \item bijectie: 1-op-1-afbeelding: voor elke x 1 (x,y) en voor elke y 1 (x,y)
    \item transformatie: afbeelding als A = B: $f: A \rightarrow A$ transformatie asa elke x $\in$ A beeld  heeft onder f
    \item permutatie: bijectie als A = B: $f: A \rightarrow A$ permutatie asa elke x $\in$ A beeld onder f en precies 1 beeld x onder f
\end{itemize}

\subsection{Functie met meerdere argumenten: f(x,y,...)}
voor elke tupel $(x_{1},x_{2},...,x_{n})$ hoogstens 1 y $\in$ B zo dat $(x_{1},x_{2},...,x_{n},y) \in R$

\subsection{Operatoren op functies}
\begin{itemize}
    \item samenstelling van $f: A \rightarrow B$ en $g: B \rightarrow C$ = $h: A \rightarrow C : x \mapsto g(f(x))$ \newline andere notatie: g $\circ$ f, als f $\circ$ f = $f^2$
    \item inverse functie $f^{-1}: B \rightarrow A$ met $f(x) = y \iff f^{-1} = x$ 
    \newline $\Rightarrow$ moet voldaan: : $f: A \rightarrow B$ met $\forall x \in A: f(x) = f(x') \Rightarrow x=x'$ 
    \newline $\Rightarrow$ inverteerbaar
    \item identiteitsfunctie: $i_{A} : A \rightarrow A : x \mapsto x$ 
    \newline $\Rightarrow$ zij f : A $\rightarrow$ B inverteerbare afbeelding dan $f^{-1} \circ f = i_{A}$
\end{itemize}

\section{Equivalentierelaties en orderelaties}
\subsection{Binaire relaties over dezelfde verzameling:} \quad $R \subseteq A \times A$ 
\begin{itemize}
    \item Reflexiviteit: 
    \begin{itemize}
        \item $R \subseteq A \times A$ is reflexief asa $\forall x \in A:(x,x) \in R$
        \item $R \subseteq A \times A$ is anti-reflexief asa $\forall x \in A:(x,x) \notin R$
    \end{itemize}
    vb: reflexief: $\leq$ want x $\leq x$ \newline
    vb: anti-reflexief: $<$ want x $<$ x voor geen enkele x
    \item Symmetrie:
    \begin{itemize}
        \item $R \subseteq A \times A$ symmetrisch asa $\forall x,y \in A:(x,y) \in R \iff (y,x) \in R$
        \item $R \subseteq A \times A$ anti-symmetrisch asa $\forall x,y \in A:(x,y) \in R \land (y,x) \in R \Rightarrow x = y$
    \end{itemize}
    \item Transiviteit: $R \subseteq A \times A$ transitief asa $\forall x,y,z \in A:(x,y) \in R \land (y,z) \in R \Rightarrow (x,z) \in R$
\end{itemize}
\subsection{Equivalentierelaties}
= een relatie die reflexief, symmetrisch en transitief is. Symbool: $\sim$
\begin{itemize}
    \item Equivalentieklasse van x $\in$ A en A heeft equivalentierelatie
    \newline \quad = K(x): verzameling van alle elementen van A die equivalent zijn met x
    \newline \quad K(x) = $\{y \in A| x \sim y\}$
    \newline eig:
    \begin{itemize}
        \item alle elementen in een equivalentieklasse zijn equivalent met elkaar
        \item Als $x \sim y$, is K(x) = K(y)
        \item Als $x \not \sim y$, is K(x) $\cap$ K(y) = $\emptyset$
    \end{itemize}
    \item Partitie: P = $\{P_1,P_2,...,P_k\}$ is een partitie van A asa
    \begin{itemize}
        \item $\forall x \in A: \exists ! P_i \in P: x \in P_i$
        \item $\forall P_i \in P: P_i \neq \emptyset$
    \end{itemize}
    \item Quotiëntverzameling (= de partitie van A die bestaat ui alle equivalentieklassen van A onder $\sim$) A/$\sim$ : "A onder $\sim$": A heeft equivalentierelatie $\sim$
\end{itemize}
\subsection{Orderelaties}
= relatie die reflexief, antisymmetrisch en transitief is
\begin{itemize}
    \item[$*$] voor willekeurige orderelatie: $x \preceq y$: "x komt voor of is gelijk aan y"
    \item[$*$] geordende verzameling A,$\preceq$
\end{itemize}
\begin{itemize}
    \item totale orde: orderelatie waarvoor geldt: $\forall x,y: x \preceq y \lor y \preceq x$, anders: partiële orde
    \item bovengrens voor X: $a \in A$ als $\forall x \in X: x \preceq a$
    \item ondergrens voor X: $a \in A$ als $\forall x \in X: a \preceq x$
    \item supremum van X: bovengrens b van X zodat b $\preceq$ b' voor alle bovengrenzen b' van X
    \newline = kleinste bovengrens voor X, als die bestaat
    \item infimum van X: ondergrens o van X zodat o' $\preceq$ o voor alle ondergrenzen o' van X
    \newline = grootste ondergrens voor X, als het bestaat
\end{itemize}
\subsection{Tralies}
\begin{itemize}
    \item complete tralie: geordende verzameling A,$\preceq$ met de eigenschap dat elke eindige verzameling X $\subseteq$ A een supremum en infumum in A heeft
    \item grafish \newline tralie van P({a,b,c}),$\subseteq$ \newline \includegraphics{tralie} \newline
    tralie van $\mathbb{N} ,\leq$
    \newline \quad 0 - 1 - 2 - 3 - 4 - 5 - ...
    \item voor eender welke verzameling U geldt: P(U),$\subseteq$ is een complete tralie
\end{itemize}
\subsection{Quasi-ordes}
= relatie die reflexief en transitief is
$\Rightarrow$ veralgemeent equivalentierelaties als orderelaties
\begin{itemize}
    \item symmetrische quasi-orde = equivalentierelatie
    \item anti-symmetrische quasi-orde = orderelatie
\end{itemize}

\section{Oefeningen}
\href{https://docs.google.com/spreadsheets/d/1thtWaGEHPFeVuYDWK-XKuEimLSO2Zc_yDwdODAbYIok/edit#gid=0}{Google Spreadsheet}
\end{document}
